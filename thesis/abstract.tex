% abstract
\thispagestyle{plain} % no header
\centerline{\bfseries Abstract}
\vspace*{1em}
\noindent

The \ac{RPL} has evolved into one of the most commonly used routing protocols
for \acp{WSN}. It has already been evaluated under various criteria and for a
wide range of applications. Previous work has established that single or
multiple node restarts can transiently affect the rest of the network. The
implications of this effect motivated a consecutive work that implemented a
hardened version of \ac{RPL} based on the \emph{Contiki} operating system. The
implementation has then been evaluated both using the \emph{Cooja} network
simulator and a small test network. It has been concluded that the hardened
implementation significantly reduced the control message overhead for \ac{WSN}
in which single or multiple node resets occur.

This thesis builds upon these findings by evaluating the effect of transient
node failures using \fitlab, a large scale \ac{IoT} test network. For this
purpose software has been created which automates the execution of the
experiments and automatically collects and analyses the results.

The general effect of transient node failures on \ac{RPL} has been confirmed
using \fitlab. The hardened implementation manages to significantly reduce the
control message overhead of a recovering network, but increases the overhead if
no resets occur. Other criteria, such as the packet delivery ratio and the
network convergence time, are adversely affected by it.

\acresetall

%\vspace*{5em}
%
%\centerline{\bfseries Zusammenfassung}
%\vspace*{1em}
%\noindent
%Hier steht die deutsche Zusammenfassung, bei Arbeiten auf Deutsch steht die deutsche Zusammenfassung oben. If your thesis is in english, the german Zusammenfassung is optional.
\cleardoublepage